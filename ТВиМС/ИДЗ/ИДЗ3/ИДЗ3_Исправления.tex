\input{D:/GitHub/currentWork/TeX_preambles/preamble.tex}

\title{Теория вероятностей и мат. статистика \\ ИДЗ3}
\date{04.05.2020}
\author{Почаев Никита Алексеевич, гр. 8381 \\ \href{mailto:pochaev.nik@gmail.com}{pochaev.nik@gmail.com} \\ Преподаватель: Малов Сергей Васильевич}

\begin{document}
	
\renewcommand{\figurename}{Рисунок}

\maketitle

\begin{figure}[H]
	\center{\includegraphics[scale=1]{./media/Задание.pdf}}
\end{figure}
	
\[ E\xi = \int_{-\infty}^{\infty} x \cdot p_{\xi} (x) dx = \int_{1}^{5} x \left( \frac{x-1}{8} \right) dx = \frac{11}{3}; ~~~~~~ E\xi^2 = \int_{1}^{5} x^2 \left( \frac{x-1}{8} \right) dx = \frac{43}{3} \]
\[ E\eta = \int_{-\infty}^{\infty} y \cdot p_{\eta} (y) dy = \int_{-1}^{1} y \left( \frac{1-y}{2} \right) dy = -\frac{1}{3}; ~~~~~~ E\eta^2 = \int_{-1}^{1} y^2 \left( \frac{1-y}{2} \right) dy = \frac{1}{3} \]
\[ D\xi = E\xi^2 - (E\xi)^2 = \frac{43}{3} - \left(\frac{11}{3}\right)^2 = \frac{8}{9}; ~~~~~~ D\eta = E\eta^2 - (E\eta)^2 = \frac{1}{3} - \left(- \frac{1}{3}\right)^2 = \frac{2}{9} \]
Вектор мат. ожидания: $E = \begin{pmatrix} \xi \\ \eta \end{pmatrix} = \frac{1}{3} \begin{pmatrix} 11 \\ -1 \end{pmatrix}$
\[ E\xi\eta = \int_{\mathbb{R}^2} xy \cdot p_{\xi, \eta} (x,y) dxdy = \int_{1}^{5} dx \int_{-1}^{\frac{x-3}{2}} xy \cdot \frac{1}{4} dy = \int_{1}^{5} \frac{1}{32} x (x^2-6x+5) dx = -1 \]
\begin{center}
	\fbox{%
		\parbox[t][6.5cm]{13cm}{%
			Пусть $\eta, \xi$ - две случайные величины, определённые на одном и том же вероятностном пространстве. Тогда ковариацией случайных величин (\textit{англ.} covariance) $\eta$ и $\xi$ называется выражение следующего вида:
			\[ \cov (\eta, \xi) = E((\eta - E\eta) \cdot (\xi - E\xi)) \]
			В силу линейности математического ожидания, ковариация может быть записана как:
			\[ \cov (\eta, \xi) = E (\xi \cdot \eta - \eta \cdot E\xi + E\xi \cdot E\eta - \xi \cdot E\eta) = \]
			\[ = E(\xi \cdot \eta) - E\xi \cdot E\eta - E\xi \cdot E\eta + E\xi \cdot E\eta = \]
			\[ = E(\xi \cdot \eta) - E\xi \cdot E\eta \]
	}}\qquad
\end{center}
\[ \cov (\xi, \eta) = -1 + \frac{11}{3} \cdot \frac{1}{3} = \frac{2}{9} \]
Коэффициент корреляции:
\[ r(\xi, \eta) = \frac{\cov (\xi, \eta)}{\sqrt{D\xi \cdot D\eta}} = \frac{\frac{2}{9}}{\sqrt{\frac{8}{9} \cdot \frac{2}{9}}} = \frac{1}{2} \]
Матрица корреляции:
\[ \corr \begin{pmatrix} \xi \\ \eta \end{pmatrix} = \begin{pmatrix} 1 & \frac{1}{2} \\ \frac{1}{2} & 1 \end{pmatrix} \]
\begin{center}
	\fbox{%
		\parbox[t][3.5cm]{13cm}{%
			Матрица ковариаций (\textit{англ.} covariance matrix) — это матрица, элементы которой являются попарными ковариациями элементов одного или двух случайных векторов. Ковариационная матрица случайного вектора — квадратная симметрическая неотрицательно определенная матрица, на диагонали которой располагаются дисперсии компонент вектора, а внедиагональные элементы — ковариации между компонентами.
	}}\qquad
\end{center}
Матрица ковариации:
\[ \var \begin{pmatrix} \xi \\ \eta \end{pmatrix} = \begin{pmatrix} D\xi & \cov (\xi, \eta) \\ \cov (\xi, \eta) & D\eta \end{pmatrix} = \begin{pmatrix} \frac{8}{9} & \frac{2}{9} \\ \frac{2}{9} & \frac{2}{9} \end{pmatrix} \]
Условное распределение $\xi$ при условии $\eta$:
\[
p_{\xi|\eta = y_0} (x) = \frac{p_{\xi, \eta}  (x,y)}{p_{\eta} (y)} =
\begin{cases}
	\frac{1}{2(1-y_0)}, x \le 5, y \ge -1, 2x - 4y \ge 6 \\
	0 - \text{ в остальных случаях}
\end{cases}
\]
\begin{center}
	\fbox{%
		\parbox[t][3cm]{8cm}{%
			\[ f_{\xi | \eta} (x | y_0) \ge 0 \text{ почти всюду на } \mathbb{R}^{m+n} \]
			\[ \int_{\mathbb{R}^m} f_{\xi | \eta} (x | y_0) dx = 1, \forall y_0 \in \mathbb{R}^n \]
	}}\qquad
\end{center}
Проверим корректность вычислений, зафиксировав $y=0 \Rightarrow x = 3$: $\int\limits_{3}^{5} \dfrac{1}{2(1-0)}dx = 1$.
\[
p_{\xi | \eta} (x) =
\begin{cases}
	\frac{1}{2(1-\eta)}, \text{ при } x \le 5, 2x - 4 \eta \ge 6 \\
	0 - \text{ в остальых случаях}
\end{cases}
\]
Условное мат. ожидание:
\[ E(\xi | \eta = y_0) = \int_{1}^{5} x \cdot p_{\xi | \eta = y_0} (x) dx = \int_{1}^{5} \frac{x}{2(1-y_0)} dx = -\frac{6}{y-1} \Rightarrow E(\xi|\eta) = \frac{-6}{\eta-1} \]
\[ E(\xi^2 | \eta = y_0) = \int_{1}^{5} x^2 \cdot p_{\xi | \eta = y_0} (x) dx = \int_{1}^{5} \frac{x^2}{2(1-y_0)} dx = \frac{62}{3-3y_0} \]
\[ D(\xi | \eta = y_0) = \frac{62}{3-3y_0} - \left( -\frac{6}{y_0-1} \right)^2 = -\frac{2(31y_0 + 23)}{3(y_0-1)^2} \Rightarrow D(\xi | \eta) = -\frac{2(31\eta + 23)}{3(\eta-1)^2} \]
	

\[
F_{\mu} (z) = P(-4\xi + 8 \eta < z) =
\begin{cases}
	0, &z \le -28 \\
	*, &z \in (-28, -12] \\
	1, &z > -12
\end{cases}, \supp (\mu) = [-28, -12]
\]
\begin{figure}[H]
	\center{\includegraphics[scale=0.9]{./media/4_1.pdf}}
\end{figure}
При $z \in (-28, -12]: -4x + 8y = z, y = -1.5 \Rightarrow x = -\left(\dfrac{z}{4}+2\right)$.
\[ * = F_{\mu}(z) = \int_{-\left(\frac{z}{4}+2\right)}^{5} dx \int_{-1}^{\frac{z}{8} + \frac{1}{2}x} \frac{1}{4} dy = \int_{-\left(\frac{z}{4}+2\right)}^{5} \frac{1}{32} (4x + z + 8) dx = \frac{1}{256} (z + 28)^2 \]
Итак, функция распределения:
\[
F_{\mu} (z) = P(-4\xi + 8 \eta < z) =
\begin{cases}
	0, &z \le -28 \\
	\frac{1}{256} (z + 28)^2, &z \in (-28, -12] \\
	1, &z > -12
\end{cases}
\]
График функции приведён ниже:
\begin{figure}[H]
	\center{\includegraphics[scale=0.9]{./media/4_2.pdf}}
\end{figure}
\[
p_{\nu} (z) =
\begin{cases}
	\dfrac{z + 28}{128}, z \in [-28,-12] \\
	0 - \text{ в остальных случаях}
\end{cases}
\]
\[ E\mu \int_{-28}^{-12} z \cdot \frac{z + 28}{128} dx = -\frac{52}{3} \approx -17.333; ~~~~~~~~~ E\mu^2 \int_{-28}^{-12} z^2 \cdot \frac{z + 28}{128} dx = \frac{944}{3} \approx 314.667 \]
\[ D\mu^2 = E\mu^2 - (E\mu)^2 = \frac{944}{3} - \left( -\frac{52}{3} \right)^2 = \frac{128}{9} \approx 14.\bar{2} \]

\end{document} 